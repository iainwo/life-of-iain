\subsection{Excercises}

\begin{enumerate}
  \item Use the roster notation to designate the following sets of real numbers.
  
  \begin{proposition}
    $A = \{ x | x^2 - 1 = 0\}$ can be designated as $\{-1, 1\}$ in roster notation.
  \end{proposition}
  \begin{proof}
    \begin{equation}
      \begin{split}
        A & = \{x | x^2 - 1 = 0\} \\
        & = \{x | (x - 1)(x + 1) = 0\} \\
        & \therefore \{-1, 1\}
      \end{split}
    \end{equation}
  \end{proof}

  \begin{proposition}
    $B = \{x | (x-1)^2 = 0\}$ can be designated as $\{1\}$ in roster notation.
  \end{proposition}
  \begin{proof}
    \begin{equation}
      \begin{split}
        B & = \{x | (x-1)^2 = 0\} \\
        & = \{x | x-1 = \sqrt{0} \} \\
        & = \{x | x = 1 \} \\
        & \therefore \{1\}
      \end{split}
    \end{equation}
  \end{proof}

  \begin{proposition}
    $C = \{x | x + 8 = 9\}$ can be designated as $\{ 1 \}$ in roster notation.
  \end{proposition}
  \begin{proof}
    \begin{equation}
      \begin{split}
        C & = \{x | x + 8 = 9\} \\
        & = \{x | x = 9 - 8 \} \\
        & = \{x | x = 1 \} \\
        & \therefore \{ 1 \}
      \end{split}
    \end{equation}
  \end{proof}

  \begin{proposition}
    $D = \{x | x^3 - 2x^2 + x = 2\}$ can be designated as $\{2\}$ in roster notation.
  \end{proposition}
  \begin{proof}
    \begin{equation}
      \begin{split}
        D & = \{x | x^3 - 2x^2 + x = 2\} \\
        & = \{x | x^3 - 2x^2 + x - 2 = 0 \} \\
        & = \{x | x^2(x - 2) + (x - 2) = 0 \} \\
        & = \{x | (x^2 + 1)(x - 2) = 0 \} \\
        & \therefore \{2\}
      \end{split}
    \end{equation}
  \end{proof}

  \begin{proposition}
    $E = \{x | (x + 8)^2 = 9^2 \}$ can be designated as $\{-17, 1\}$ in roster notation.
  \end{proposition}
  \begin{proof}
    \begin{equation}
      \begin{split}
        E & = \{x | (x + 8)^2 = 9^2 \} \\
        & = \{x | x + 8 = \pm{9} \} \\
        & = \{x | x = \pm{9} - 8 \} \\
        & \therefore \{-17, 1\}
      \end{split}
    \end{equation}
  \end{proof}

  \begin{proposition}
    $F = \{x | (x^2 + 16x)^2 = 17^2 \}$ can be designated as $\{-17, 1, -8 - \sqrt{47}, -8 + \sqrt{47}\}$ in roster notation.
  \end{proposition}
  \begin{proof}
    \begin{equation}
      \label{eq:some_nontrivial_root}
      \begin{split}
        F & = \{x | (x^2 + 16x)^2 = 17^2 \} \\
        & = \{x | x^2 + 16x = \pm{17} \} \\
        & = \{x | x^2 + 16x \pm{17} = 0 \} \\
        & = \{x | x^2 + 16x \pm{17} = 0 \} \\
      \end{split}
    \end{equation}
    
    Using the quadratic formula:
    \begin{definition}{Quadratic Equation}, analytical method for calculating the roots of a quadratic polynomial.
      \begin{equation}
        \begin{split}
          x & = \frac{-b \pm{\sqrt{b^2 - 4ac}}}{2a} \text{, where } ax^2 + bx + c = 0
        \end{split}
      \end{equation}
    \end{definition}
    
    Solving when the last term is $+17$:
    \begin{equation}
      \begin{split}
        x & = \frac{-16 \pm{\sqrt{16^2 - 4(1)17}}}{2(1)} \\
        & = -8 \pm{\frac{\sqrt{188}}{2}} \\
        & = -8 \pm{\frac{\sqrt{188}}{\sqrt{2^2}}} \\
        & = -8 \pm{\sqrt{188/4}} \\
        & = -8 \pm{\sqrt{47}} \\
        & \therefore \{-8 - \sqrt{47}, -8 + \sqrt{47}\}
      \end{split}
    \end{equation}

    Solving when the last term is $-17$:
    \begin{equation}
      \begin{split}
        0 & = x^2 + 16x - 17 \\
        & = (x + 17)(x - 1) \\
        & \therefore \{-17, 1\}
      \end{split}
    \end{equation}
    
  \end{proof}

  \item For the sets in Exercise 1, note that $B \subseteq A$. List all the inclusion relations $\subseteq$ that hold among the sets $A$, $B$, $C$, $D$, $E$, $F$.
  \begin{enumerate}
    \item $A \subseteq A$
    \item $B \subseteq A$
    \item $B \subseteq B$
    \item $B \subseteq C$
    \item $B \subseteq E$
    \item $B \subseteq F$
    \item $C \subseteq A$
    \item $C \subseteq B$
    \item $C \subseteq C$
    \item $C \subseteq E$
    \item $C \subseteq F$
    \item $D \subseteq D$
    \item $E \subseteq E$
    \item $E \subseteq F$
    \item $F \subseteq F$
  \end{enumerate}

  \item Let $A = \{1\}$, $B = \{1, 2\}$. Discuss the validity of the following statements (prove the ones that are true and explain why the others are not true).
  \begin{definition}{Set Equality}
    Two sets $A$ and $B$ are said to be equal (or identical) if they consisit of exactly the same elements, in which case we write $A = B$. If one of the sets contains an element not in the other, we say the sets are unequal and we write $A \neq B$.
  \end{definition}
  \begin{definition}{Subset}
    \label{definition:subset}
    A set $A$ is said to be a subset of a set $B$, and we write $A \subseteq B$ whenever every element of $A$ also belongs to $B$. We also say that $A$ is contained in $B$ or that $B$ contains $A$. The relation $\subseteq$ is referred to as set inclusion.
  \end{definition}
  \begin{enumerate}

    \item \begin{proposition}
      $A \subset B$
    \end{proposition}
    \begin{proof}
      \begin{equation}
        \begin{split}
          \{ x \in A | \exists y \in B (x = y)\}
        \end{split}
      \end{equation}
    \end{proof}

    \item \begin{proposition}
      $A \subseteq B$
    \end{proposition}
    \begin{proof}
      \begin{equation}
        \begin{split}
          \{ x \in A | \exists y \in B (x = y)\} \text{, by the definition of a subset \ref{definition:subset}}
        \end{split}
      \end{equation}
    \end{proof}

    \item \begin{proposition}
      $A \in B$
    \end{proposition}
    \begin{proof}
      \begin{equation}
        \begin{split}
          & \forall x \in B: x \neq A \\
          & \therefore A \notin B
        \end{split}
      \end{equation}
    \end{proof}

    \item \begin{proposition}
      $1 \in A$
    \end{proposition}
    \begin{proof}
      \begin{equation}
        \begin{split}
          \exists x \in A(x = 1)
        \end{split}
      \end{equation}
    \end{proof}

    \item \begin{proposition}
      $1 \subseteq A$
    \end{proposition}
    \begin{proof}
      \begin{equation}
        \begin{split}
          & \forall x \in \mathcal{P(A)}: 1 \neq x \text{, where } \mathcal{P(A)} \text{ is the powerset of A and x each subset} \\
          & \therefore 1 \not \subset A
        \end{split}
      \end{equation}
    \end{proof}

    \item \begin{proposition}
      $1 \subset B$
    \end{proposition}
    \begin{proof}
      \begin{equation}
        \begin{split}
          & \forall x \in \mathcal{P(B)}: 1 \neq x \text{, where } \mathcal{P(B)} \text{ is the powerset of B and x each subset} \\
          & \therefore 1 \not \subset B
        \end{split}
      \end{equation}
    \end{proof}
  \end{enumerate}

  \item Solve the previous exercise if $A = \{1\}$ and $B = \{ \{1 \}, 1 \}$.
  \begin{enumerate}
    \item \begin{proposition}
      $A \subset B$
    \end{proposition}
    \begin{proof}
      \begin{equation}
        \begin{split}
          (\O \neq (A \cap B)) \land ((A \cap B) \subset B)
        \end{split}
      \end{equation}
    \end{proof}
  \end{enumerate}

  \item Given the set $S = \{1, 2, 3, 4\}$. Display all subsets of $S$. There are $16$ altogether, counting $\O$ and $S$.
  \begin{equation}
    \begin{split}
      \mathcal{P(S)} = \bigcup\limits_{i=1}^{|S|} \bigcup\limits_{j=1}^{|S|-i+1} \{S_{i}, ..., s_{j}\} \cup \{\O\}
    \end{split}
  \end{equation}

  \item Given the following four sets
  \begin{math}
    A = \{1, 2\}, B = \{\{1\}, \{2\}\}, C = \{\{1\},\{1, 2\}\}, D = \{\{1\}, \{2\}, \{1, 2\}\}
  \end{math}
  discuss the validty of the following statements (prove the ones that are true and explain why the others are not true).
  \begin{enumerate}
    \item \begin{proposition}
      A = B
    \end{proposition}
    \begin{proof}
      \begin{equation}
        \begin{split}
          & \exists x \in A: x \notin B \\
          & \therefore A \neq B
        \end{split}
      \end{equation}
    \end{proof}

    \item \begin{proposition}
      $A \subseteq B$
    \end{proposition}
    \begin{proof}
      \begin{equation}
        \begin{split}
          & \forall x \in A: x \notin B \\
          & \therefore A \nsubseteq B
        \end{split}
      \end{equation}
    \end{proof}

    \item \begin{proposition}
      $A \subset C$
    \end{proposition}
    \begin{proof}
      \begin{equation}
        \begin{split}
          & \forall x \in A: x \notin C \\
          & \therefore A \not \subset C
        \end{split}
      \end{equation}
    \end{proof}

    \item \begin{proposition}
      $A \in C$
    \end{proposition}
    \begin{proof}
      \begin{equation}
        \begin{split}
          & \O \neq (\{A\} \cap C) \\
          & \therefore A \in C
        \end{split}
      \end{equation}
    \end{proof}

  \end{enumerate}
\end{enumerate}