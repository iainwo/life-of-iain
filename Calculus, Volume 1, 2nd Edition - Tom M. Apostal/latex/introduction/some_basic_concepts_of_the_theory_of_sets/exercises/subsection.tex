\setcounter{subsection}{4}
\subsection{Excercises}

\begin{question} Use the roster notation to designate the following sets of real numbers.
  \begin{subquestion}
    $A = \{ x | x^2 - 1 = 0\}$ can be designated as $\{-1, 1\}$ in roster notation.
  \end{subquestion}
  \begin{proof}
    \begin{equation}
      \begin{split}
        A & = \{x | x^2 - 1 = 0\} \\
        & = \{x | (x - 1)(x + 1) = 0\} \\
        & \therefore \{-1, 1\}
      \end{split}
    \end{equation}
  \end{proof}

  \begin{subquestion}
    $B = \{x | (x-1)^2 = 0\}$ can be designated as $\{1\}$ in roster notation.
  \end{subquestion}
  \begin{proof}
    \begin{equation}
      \begin{split}
        B & = \{x | (x-1)^2 = 0\} \\
        & = \{x | x-1 = \sqrt{0} \} \\
        & = \{x | x = 1 \} \\
        & \therefore \{1\}
      \end{split}
    \end{equation}
  \end{proof}

  \begin{subquestion}
    $C = \{x | x + 8 = 9\}$ can be designated as $\{ 1 \}$ in roster notation.
  \end{subquestion}
  \begin{proof}
    \begin{equation}
      \begin{split}
        C & = \{x | x + 8 = 9\} \\
        & = \{x | x = 9 - 8 \} \\
        & = \{x | x = 1 \} \\
        & \therefore \{ 1 \}
      \end{split}
    \end{equation}
  \end{proof}

  \begin{subquestion}
    $D = \{x | x^3 - 2x^2 + x = 2\}$ can be designated as $\{2\}$ in roster notation.
  \end{subquestion}
  \begin{proof}
    \begin{equation}
      \begin{split}
        D & = \{x | x^3 - 2x^2 + x = 2\} \\
        & = \{x | x^3 - 2x^2 + x - 2 = 0 \} \\
        & = \{x | x^2(x - 2) + (x - 2) = 0 \} \\
        & = \{x | (x^2 + 1)(x - 2) = 0 \} \\
        & \therefore \{2\}
      \end{split}
    \end{equation}
  \end{proof}

  \begin{subquestion}
    $E = \{x | (x + 8)^2 = 9^2 \}$ can be designated as $\{-17, 1\}$ in roster notation.
  \end{subquestion}
  \begin{proof}
    \begin{equation}
      \begin{split}
        E & = \{x | (x + 8)^2 = 9^2 \} \\
        & = \{x | x + 8 = \pm{9} \} \\
        & = \{x | x = \pm{9} - 8 \} \\
        & \therefore \{-17, 1\}
      \end{split}
    \end{equation}
  \end{proof}

  \begin{subquestion}
    $F = \{x | (x^2 + 16x)^2 = 17^2 \}$ can be designated as $\{-17, 1, -8 - \sqrt{47}, -8 + \sqrt{47}\}$ in roster notation.
  \end{subquestion}
  \begin{proof}
    \begin{equation}
      \label{eq:some_nontrivial_root}
      \begin{split}
        F & = \{x | (x^2 + 16x)^2 = 17^2 \} \\
        & = \{x | x^2 + 16x = \pm{17} \} \\
        & = \{x | x^2 + 16x \pm{17} = 0 \} \\
        & = \{x | x^2 + 16x \pm{17} = 0 \} \\
      \end{split}
    \end{equation}
    
    Using the quadratic formula:
    \begin{definition}{Quadratic Equation}, analytical method for calculating the roots of a quadratic polynomial.
      \begin{equation}
        \begin{split}
          x & = \frac{-b \pm{\sqrt{b^2 - 4ac}}}{2a} \text{, where } ax^2 + bx + c = 0
        \end{split}
      \end{equation}
    \end{definition}
    
    Solving when the last term is $+17$:
    \begin{equation}
      \begin{split}
        x & = \frac{-16 \pm{\sqrt{16^2 - 4(1)17}}}{2(1)} \\
        & = -8 \pm{\frac{\sqrt{188}}{2}} \\
        & = -8 \pm{\frac{\sqrt{188}}{\sqrt{2^2}}} \\
        & = -8 \pm{\sqrt{188/4}} \\
        & = -8 \pm{\sqrt{47}} \\
        & \therefore \{-8 - \sqrt{47}, -8 + \sqrt{47}\}
      \end{split}
    \end{equation}

    Solving when the last term is $-17$:
    \begin{equation}
      \begin{split}
        0 & = x^2 + 16x - 17 \\
        & = (x + 17)(x - 1) \\
        & \therefore \{-17, 1\}
      \end{split}
    \end{equation}
    
  \end{proof}
\end{question}

\begin{question} For the sets in Exercise 1, note that $B \subseteq A$. List all the inclusion relations $\subseteq$ that hold among the sets $A$, $B$, $C$, $D$, $E$, $F$.
  \begin{enumerate}
    \item $A \subseteq A$
    \item $B \subseteq A$
    \item $B \subseteq B$
    \item $B \subseteq C$
    \item $B \subseteq E$
    \item $B \subseteq F$
    \item $C \subseteq A$
    \item $C \subseteq B$
    \item $C \subseteq C$
    \item $C \subseteq E$
    \item $C \subseteq F$
    \item $D \subseteq D$
    \item $E \subseteq E$
    \item $E \subseteq F$
    \item $F \subseteq F$
  \end{enumerate}
\end{question}


\begin{question} Let $A = \{1\}$, $B = \{1, 2\}$. Discuss the validity of the following statements (prove the ones that are true and explain why the others are not true).
  \begin{definition}{Set Equality}
    \label{definition:set_equality}
    Two sets $A$ and $B$ are said to be equal (or identical) if they consisit of exactly the same elements, in which case we write $A = B$. If one of the sets contains an element not in the other, we say the sets are unequal and we write $A \neq B$.
  \end{definition}
  \begin{definition}{Subset}
    \label{definition:subset}
    A set $A$ is said to be a subset of a set $B$, and we write $A \subseteq B$ whenever every element of $A$ also belongs to $B$. We also say that $A$ is contained in $B$ or that $B$ contains $A$. The relation $\subseteq$ is referred to as set inclusion.
  \end{definition}

  \begin{subquestion}
    $A \subset B$
  \end{subquestion}
  \begin{proof}
    \begin{equation}
      \begin{split}
        \{ x \in A | \exists y \in B (x = y)\}
      \end{split}
    \end{equation}
  \end{proof}

  \begin{subquestion}
    $A \subseteq B$
  \end{subquestion}
  \begin{proof}
    \begin{equation}
      \begin{split}
        \{ x \in A | \exists y \in B (x = y)\} \text{, by the definition of a subset \ref{definition:subset}}
      \end{split}
    \end{equation}
  \end{proof}

  \begin{subquestion}
    $A \in B$
  \end{subquestion}
  \begin{proof}
    \begin{equation}
      \begin{split}
        & \forall x \in B: x \neq A \\
        & \therefore A \notin B
      \end{split}
    \end{equation}
  \end{proof}

  \begin{subquestion}
    $1 \in A$
  \end{subquestion}
  \begin{proof}
    \begin{equation}
      \begin{split}
        \exists x \in A(x = 1)
      \end{split}
    \end{equation}
  \end{proof}

  \begin{subquestion}
    $1 \subseteq A$
  \end{subquestion}
  \begin{proof}
    \begin{equation}
      \begin{split}
        & \forall x \in \mathcal{P(A)}: 1 \neq x \text{, where } \mathcal{P(A)} \text{ is the powerset of A and x each subset} \\
        & \therefore 1 \not \subset A
      \end{split}
    \end{equation}
  \end{proof}

  \begin{subquestion}
    $1 \subset B$
  \end{subquestion}
  \begin{proof}
    \begin{equation}
      \begin{split}
        & \forall x \in \mathcal{P(B)}: 1 \neq x \text{, where } \mathcal{P(B)} \text{ is the powerset of B and x each subset} \\
        & \therefore 1 \not \subset B
      \end{split}
    \end{equation}
  \end{proof}
\end{question}

\begin{question}
Solve the previous exercise if $A = \{1\}$ and $B = \{ \{1 \}, 1 \}$.
  \begin{subquestion}
    $A \subset B$
  \end{subquestion}
  \begin{proof}
    \begin{equation}
      \begin{split}
        (\O \neq (A \cap B)) \land ((A \cap B) \subset B)
      \end{split}
    \end{equation}
  \end{proof}
\end{question}
\begin{question}
Given the set $S = \{1, 2, 3, 4\}$. Display all subsets of $S$. There are $16$ altogether, counting $\O$ and $S$.
  \begin{equation}
    \begin{split}
      \mathcal{P(S)} = \bigcup\limits_{i=1}^{|S|} \bigcup\limits_{j=1}^{|S|-i+1} \{S_{i}, ..., s_{j}\} \cup \{\O\}
    \end{split}
  \end{equation}
\end{question}

\begin{question}x
  Given the following four sets
  \begin{math}
    A = \{1, 2\}, B = \{\{1\}, \{2\}\}, C = \{\{1\},\{1, 2\}\}, D = \{\{1\}, \{2\}, \{1, 2\}\}
  \end{math}
  discuss the validty of the following statements (prove the ones that are true and explain why the others are not true).

  \begin{subquestion}
    A = B
  \end{subquestion}
  \begin{proof}
    \begin{equation}
      \begin{split}
        & \exists x \in A: x \notin B \\
        & \therefore A \neq B
      \end{split}
    \end{equation}
  \end{proof}

  \begin{subquestion}
    $A \subseteq B$
  \end{subquestion}
  \begin{proof}
    \begin{equation}
      \begin{split}
        & \forall x \in A: x \notin B \\
        & \therefore A \nsubseteq B
      \end{split}
    \end{equation}
  \end{proof}

  \begin{subquestion}
    $A \subset C$
  \end{subquestion}
  \begin{proof}
    \begin{equation}
      \begin{split}
        & \forall x \in A: x \notin C \\
        & \therefore A \not \subset C
      \end{split}
    \end{equation}
  \end{proof}

  \begin{subquestion}
    $A \in C$
  \end{subquestion}
  \begin{proof}
    \begin{equation}
      \begin{split}
        & \O \neq (\{A\} \cap C) \\
        & \therefore A \in C
      \end{split}
    \end{equation}
  \end{proof}

  \begin{subquestion}
    $A \subset D$
  \end{subquestion}
  \begin{proof}
    \begin{equation}
      \begin{split}
        & \exists x \in A (x \not \in D) \\
        & \therefore A \not \subset D
      \end{split}
    \end{equation}
  \end{proof}

  \begin{subquestion}
    $B \subset C$
  \end{subquestion}
  \begin{proof}
    \begin{align*}
      & \exists x \in B (x \not \in C) \\
      & \therefore B \not \subset C
    \end{align*}
  \end{proof}

  \begin{subquestion}
    $B \subset D$
  \end{subquestion}
  \begin{proof}
    \begin{align*}
      & \forall x \in B (x \in D) \\
      & \therefore B \subset D
    \end{align*}
  \end{proof}

  \begin{subquestion}
    $B \in D$
  \end{subquestion}
  \begin{proof}
    \begin{align*}
      & \forall x \in D (x \neq B) \\
      & \therefore B \not \in D
    \end{align*}
  \end{proof}

  \begin{subquestion}
    $A \in D$
  \end{subquestion}
  \begin{proof}
    \begin{align*}
      & \exists x \in D (x = A) \\
      & \therefore A \in D
    \end{align*}
  \end{proof}

\end{question}

\begin{question}
  Prove the following properties of set equality.

  \begin{subquestion}
    $\{a, a\} = \{ a \}$.
  \end{subquestion}
  \begin{proof}
    \begin{align*}
      & \text{Every idiosyncracy is shared which can only be true of equivalent objects,} \\
      & \forall x \in \{a, a\} \cup \{ a \} [x \in (\{a, a\} \cap \{ a \})] \\
      & \text{Since no one set contains an element not in the other} \\
      & \text{these sets can only be equal; by the} \\
      & \text{Definition of Set Equality \ref{definition:set_equality}} \\
      & \therefore \{a, a\} = \{ a \}
    \end{align*}
  \end{proof}

  \begin{subquestion}
    $\{a, b\} = \{b, a\}$.
  \end{subquestion}
  \begin{proof}
    \begin{align*}
      & \forall x \in \{a, b\}(x \in \{b, a\}) \land \forall x \in \{b, a\}(x \in \{a, b\}) \\
      & \text{Since no one set contains an element not in the other} \\
      & \text{these sets can only be equal; by the} \\
      & \text{Definition of Set Equality \ref{definition:set_equality}} \\
      & \therefore \{a, b\} = \{b, a\}
    \end{align*}
  \end{proof}

  \begin{subquestion}
    $\{a\} = \{b, c\}$ if and only if $a = b = c$
  \end{subquestion}
  \begin{proof}
    \begin{align*}
      & \text{Let } A = \{a\} \text{ and } B = \{b, c\} \\
      & \exists x \in B (x \not \in A) \implies A \neq B \\
      & \text{Thus} \\
      & b \neq a \lor c \neq a \implies A \neq B \\
      & \text{Else} \\
      & b = a = c \implies \forall x \in B (x \in A) \land \forall x \in A (x \in B) \implies A = B
    \end{align*}
  \end{proof}

\end{question}

\begin{question}
  Commutative laws: $A \cup B = B \cup A$, $A \cap B = B \cap A$.
\end{question}
\begin{proof}
  \begin{align*}
    & \forall x \in A \cup B (x \in (B \cup A)) \land \forall x \in B \cup A (x \in (A \cup B)) \\
    & \therefore \cup \text{ is commutative.} \\
    & \text{The same proof can be applied to } \cap
  \end{align*}
\end{proof}
\begin{proof}
  Second variant. From Calculus, Volume 1, 2nd Edition.
  \begin{align*}
    & \text{Let } X = A \cup B \text{, } Y = B \cup A. \\
    & \text{To prove that } X = Y \text{ we prove that } X \subseteq Y \text{ and } Y \subseteq X. \\
    & \text{Let } x \in X. \text{Then } x \text{ is in at least one of } A \text{ or } B. \\
    & \text{Hence, } x \text{ is in at least one of } B \text{ or } A \text{; so } x \in Y. \\
    & \text{Thus, every element of } X \text{ is also in } Y \text{so } X \subseteq Y. \\
    & \text{The same can be demonstrated of } Y \subseteq X. \\
    & \therefore X = Y.
  \end{align*}
\end{proof}

\begin{question}
  Associative laws: $A \cup (B \cup C) = (A \cup B) \cup C$, $A \cap (B \cap C) = (A \cap B) \cap C$.
\end{question}
\begin{proof}
  \begin{align*}
    & \text{To prove the associative property of } \cup. \\
    & \text{We will prove the binary relation is associative} \\
    & \text{by consequence this means that any recursive composition of the operator is also associative.} \\
    & \text{Take the relationship } B \cup C. \text{ Then the relation is a superset of the members} \\ 
    & \forall x \in B(x \in (B \cup C)) \land \forall x \in C(x \in (B \cup C)) \\
    & \text{The statement above applies to any union of sets, } \\
    & \text{or union of sets formed from unions} \\
    & \text{That is this property is maintained across} \\
    & \text{ autoregressive applications of the union operator.} \\
    & \therefore A \cup (B \cup C) = (A \cup B) \cup C \\
    & \text{A similar proof ca be applied to } \cap
  \end{align*}
\end{proof}
\begin{proof}
  Second variant. From \href{http://mathonline.wikidot.com/associative-laws-of-sets}{Mathonline}.
  \begin{align*}
    & \text{Suppose that } x \in (A \cup B) \cup C. \\
    & \text{Then it follows that } x \in (A \cup B) \text{ or } x \in C. \\
    & \text{Then } x \in A \text{ or } x \in B \text{ or } x \in C. \\
    & \text{Hence, } x \in A \cup (B \cup C). \\
    & \text{Thus, } (A \cup B) \cup C \subseteq A \cup (B \cup C). \\
    & \text{The same argument can be used to show that the RHS} \subseteq \text{LHS.} \\
    & \therefore (A \cup B) \cup C = A \cup (B \cup C)
  \end{align*}
\end{proof}

\begin{question}
  Distributive laws: $A \cap (B \cup C) = (A \cap B) \cup (A \cap C)$, $A \cup (B \cap C) = (A \cup B) \cap (A \cup C)$.
\end{question}
\begin{proof}
  \begin{align*}
    & \text{For a simple set Z of atomic elements, the intersection can be computed pointwise} \\
    & Z \cap Z' = \bigcup_{x \in Z} {x} \cap Z' \\
    & \text{Let } Z = B \cup C \text{ and } Z' = A \text{ then} \\
    & \text{This pointwise algorithm can be partitioned} \\
    & \text{due to the fact the operations are pointwise/local/isolated to begin with} \\
    & Z \cap Z' = \bigcup_{x \in B \cup C} x \cap A \\
    & Z \cap Z' = (\bigcup_{x \in C} x \cap A) \cup (\bigcup_{x \in B} x \cap A) \\
    & Z \cap Z' = (B \cap A) \cup (C \cap A) \\
    & (B \cup C) \cap A = (B \cap A) \cup (C \cap A) \\
  \end{align*}
\end{proof}
\begin{proof}
  Second variant. From \href{https://math.libretexts.org/Courses/Mount_Royal_University/MATH_1150%3A_Mathematical_Reasoning/2%3A_Basic_Concepts_of_Sets/2.3%3A_Properties_of_Sets}{Pamini Thangarajah}.
  \begin{align*}
    & \text{Let } x \in A \cap (B \cup C). \\
    & \text{Then } x \in A \text{ and } x \in (B \cup C). \\
    & \text{Then } x \in A \text{ and } x \in B \text{ or } x \in C. \\
    & \text{Which implies } x \in A \text{ and } x \in B \text{ or } x \in A \text{ and } x \in C. \\
    & \text{Hence } x \in (A \cap B) \cup (A \cap C). \\
    & \text{Thus } A \cap (B \cup C) \subseteq (A \cap B) \cup (A \cap C). \\
    & \text{The same argument can be made that the RHS} \subseteq \text{LHS.} \\
    & \therefore A \cap (B \cup C) = (A \cap B) \cup (A \cap C)
  \end{align*}
\end{proof}

\begin{question}
  $A \cup A = A, A \cap A = A$
\end{question}
\begin{proof}
  The proof below can also be used for intersections by swapping out the or(s) for and(s).
  \begin{align*}
    & \text{Let } x \in A \cup A'. \\
    & \text{Then } x \in A \text{ or } x \in A'. \\
    & \text{If A' = A, then } x \in A \text{ or } x \in A. \\
    & \text{Hence, } (A \cup A) \subseteq A. \\
    & \text{The same process can be used to prove } A \subseteq (A \cup A). \\
    & \therefore (A \cup A) = A.
  \end{align*}
\end{proof}

\begin{question}
  $ A \subseteq A \cup B, A \cap B \subseteq A. $
\end{question}
\begin{proof}
  $\text{We will prove } A \subseteq A \cup B.$
  \begin{align*}
    & \text{Let } x \in A \cup B. \\
    & \text{Then } x \in A \text{ or } x \in B. \\
    & \text{Hence } x \in A \implies x \in A \cup B. \\
    & \therefore A \subseteq A \cup B.
  \end{align*}
\end{proof}
\begin{proof}
  $\text{We will prove } A \cap B \subseteq A.$
  \begin{align*}
    & \text{Let } x \in A \cap B. \\
    & \text{Then } x \in A \text{ and } x \in B. \\
    & \text{Hence, } x \in  A \cap B \implies x \in A. \\
    & \therefore A \cap B \subseteq A.
  \end{align*}
\end{proof}

\begin{question}
  $ A \cup \O = A \text{, } A \cap \O = \O $
\end{question}
\begin{proof}
  Proving $ A \cup \O = A$
  \begin{align*}
    & \text{Let } x \in A \cup \O. \\
    & \text{Then } x \in A \text{ or } x \in \O. \\
    & \text{Since } x \in \O \text{ is always false: }  x \in A \cup \O \implies x \in A. \\
    & \text{Hence, both LHS and RHS are identical by definition.} \\
    & \therefore A \cup \O = A.
  \end{align*}
\end{proof}
\begin{proof}
  Proving $A \cap \O = \O $
  \begin{align*}
    & \text{Let } x \in A \cap \O. \\
    & \text{Then } x \in A \text{ and } x \in \O. \\
    & \text{By definition the empty set } \O \text{ is empty.} \\
    & \text{Hence, } x \in \O \implies false. \\
    & \text{Hence, } x \in A \text{ and } x \in \O \implies \text{false}. \\
    & \text{Thus, } A \cap \O \text{ has no elements and is always empty}. \\
    & \therefore A \cap \O = \O
  \end{align*}
\end{proof}

\begin{question}
  $ A \cup (A \cap B) = A \text{, } A \cap (A \cup B) = A$
\end{question}
\begin{proof}
  Proving $A \cup (A \cap B) = A$. The same approach can be used for the second equivalence class.
  \begin{align*}
    & \text{By the distributive property of sets, } A \cup (A \cap B) \implies (A \cup A) \cap (A \cup B). \\
    & \text{Let } x \in (A \cup A) \cap (A \cup B). \\
    & \text{Then } x \in (A \cup A) \text{ and } x \in (A \cup B). \\
    & \text{Then } x \in A \text{ or } x \in A \text{ and } x \in A \text{ or } x \in B. \\
    & \text{Hence, } x \not \in A \implies x \not \in A \cup (A \cap B). \\
    & \text{Thus, } A \cup (A \cap B) \subseteq A. \\
    & \text{By the simplified definition above } A \subseteq A \cup (A \cap B). \\
    & \therefore A \cup (A \cap B) = A.
  \end{align*}
\end{proof}

\begin{question}
  If $ A \subseteq C$ and $B \subseteq C$, then $A \cup B \subseteq C$.
\end{question}
\begin{proof}
  \begin{align*}
    & \text{Let } x \in A \cup B. \\
    & \text{Then } x \in A \text{ or } x \in B. \\
    & \text{Then } x \in A \text{ or } x \in B \implies x \in C. \\
    & \text{Hence } x \in A \implies x \in C \text{ and } x \in B \implies x \in C. \\
    & \text{Hence } A \subseteq C \text{ and } B \subseteq C. \\
    & \therefore A \subseteq C \land B \subseteq C \implies A \cup B \subseteq C.
  \end{align*}
\end{proof}

\begin{question}
  If $C \subseteq A$ and $C \subseteq B$, then $C \subseteq A \cap B$.
\end{question}
\begin{proof}
  \begin{align*}
    & \text{Let } x \in C. \\
    & \text{If } C \subseteq A \text{ and } C \subseteq B \implies x \in A \land x \in B. \\ 
    & \therefore C \subseteq A \cap B.
  \end{align*}
\end{proof}


\begin{question}

  \begin{theorem}
    Subset Transitivity.
    If $B \subset C$ and $A \subset B$ then $A \subset C$.
    \begin{proof}
      \begin{align*}
        & \text{Let } A, B, C \text{ be sets with the relationship } A \subset B \text{ and } B \subset C. \\
        & \text{Then } A \subset B \text{ and } B \subset C \text{ can be restated } A \subset B \subset C. \\
        & \text{Hence, } x \in A \implies x \in B \implies x \in C. \\
        & \therefore A \subset C. \\
        & \text{This generalizes to improper subsets by relaxing binary "proper subset" relation to allow equivalence.}
      \end{align*}
    \end{proof}
  \end{theorem} 

  \begin{subquestion}
    If $ A \subset B$ and $B \subset C$, prove that $A \subset C$.
  \end{subquestion}
  \begin{proof}
    By the Subset Transitivity Theorem this is true.
  \end{proof}

  \begin{subquestion}
    If $A \subseteq B$ and $B \subseteq C$, prove that $A \subseteq C$.
  \end{subquestion}
  \begin{proof}
    By the Subset Transitivity Theorem this is true.
  \end{proof}

  \begin{subquestion}
    What can you conclude if $A \subset B$ and $B \subseteq C$?
  \end{subquestion}
  \begin{enumerate}
    \item $A \subset C$
    \item $A \neq C$
  \end{enumerate}

  \begin{subquestion}
    If $x \in A$ and $A \subseteq B$, is it necessarily true that $x \in B$?
  \end{subquestion}
  \begin{proof}
    \begin{align*}
      & A \subseteq B \text{ restated: } A \subset B \text{ or } A = B. \\
      & \text{Hence, } x \in A \implies x \in B. \\
      & \therefore x \in B.
    \end{align*}
  \end{proof}

  \begin{subquestion}
    If $x \in A$ and $A \in B$, is it necessarily true that $x \in B$?
  \end{subquestion}
  \begin{proof}
    \begin{align*}
      & \text{Let } B = \{ A \}. \text{ and } A = \{ \O. \} \\
      & \text{Then } A \in B. \\
      & \text{Hence, } \O \not \in B. \text { But } \{\O\} \in B. \\
      & \therefore x \in A \not \implies x \in B.
    \end{align*}
  \end{proof}

\end{question}

\begin{question}
  $A - (B \cap C) = (A - B) \cup (A - C)$.
\end{question}
\begin{proof}
  \begin{align*}
    & \text{Let } x \in A - (B \cap C). \\
    & \text{Then } x \in A \text{ and } x \not \in B \cap C. \\
    & \text{Then } x \in A \text{ and } x \not \in B \text{ and } x \not \in C. \\
    & \text{Then } x \in A \text{ and } x \not \in B \text{ and } x \in A \text{ and } x \not \in C. \\
    & \text{Hence } A - (B \cap C) \subseteq (A - B) \cup (A - C). \\
    & \text{A similar argument can be made for } (A - B) \cup (A - C). \\
    & \therefore A - (B \cap C) = (A - B) \cup (A - C).
  \end{align*}
\end{proof}

\begin{question}
  Let $F$ be a class of sets. Then $B - \bigcup\limits_{A \in F}{A} = \bigcap\limits_{A \in F}{(B-A)}$ and $B - \bigcap\limits_{A \in F}{A} = \bigcup\limits_{A \in F}{B - A}$.
\end{question}
\begin{proof}
  \begin{align*}
    & \text{Let } x \in (B - \bigcup\limits_{A \in F}{A}). \\
    & \text{Then } x \in B \text{ and } x \not \in \bigcup\limits_{A \in F}{A}. \\
    & \text{Let } C \in F. \\
    & \text{Then } x \in B \text{ and } x \not \in C \text{ and } x \in B \text{ and } x \not \in \bigcup\limits_{A \in (F - C)}{A}. \\
    & \text{Then } x \in (B - C) \text{ and } x \in (B - \bigcup\limits_{A \in (F - C)}{A}). \\
    & \text{Then } x \in (B - C) \cap x \in (B - \bigcup\limits_{A \in (F - C)}{A}). \\
    & \text{Hence, } x \in \bigcap\limits_{A \in F}{(B-A)}. \\
    & \text{Thus, } B - \bigcup\limits_{A \in F}{A} \subseteq \bigcap\limits_{A \in F}{(B-A)}. \\
    & \text{A similar argument can be made for } \bigcap\limits_{A \in F}{(B-A)} \subseteq B - \bigcup\limits_{A \in F}{A}. \\
    & \therefore B - \bigcup\limits_{A \in F}{A} = \bigcap\limits_{A \in F}{(B-A)}.
  \end{align*}
\end{proof}

\begin{question}

  \begin{subquestion}
    Prove that one of the following two formulas is always right and the other one is sometimes wrong:

    \begin{enumerate}
      \item $A - (B - C) = (A - B) \cup C$,
      \begin{proof}
        We will prove this theorem is not always true.
        \begin{align*}
          & \text{Let } x \in A - (B - C) \text{ and C = \{B, \O\}.} \\
          & \text{Then } x \in A - (\O). \\
          & \text{Then } x \in A. \\
          & \text{But the RHS evaluates to } A \cup \O. \\
          & \therefore A - (B - C) \neq (A - B) \cup C.
        \end{align*}
      \end{proof}
      \item $A - (B \cup C) = (A - B) - C$.
      \begin{proof}
        We will prove this is always true.
        \begin{align*}
          & \text{Let } x \in A - (B \cup C). \\
          & \text{Then } x \in A \text{ and } x \not \in B \cup C. \\
          & \text{Then } x \in A \text{ and } x \not \in B \text{ and } x \in A \text{ and } x \not \in C. \\
          & \text{Then } x \in A - B - C. \\
          & \text{Thus, } A - (B \cup C) \subseteq (A - B) - C. \\
          & \text{A similar argument can be made for } (A - B) - C \subseteq A - (B \cup C). \\
          & \therefore A - (B \cup C) = (A - B) - C.
        \end{align*}
      \end{proof}
    \end{enumerate}
  \end{subquestion}

  \begin{subquestion}
    State an additional necessary and sufficient condition for the formula which is sometimes incorrect to be always right.
  \end{subquestion}
  $B \subseteq C$.

\end{question}